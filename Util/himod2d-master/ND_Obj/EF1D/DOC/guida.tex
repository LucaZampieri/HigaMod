\documentclass[11pt]{article} 
\topmargin= -.5in \oddsidemargin=-0.0in \textwidth=6.5in
\textheight=9.0in 
 
\usepackage[italian]{babel} 
\usepackage[dvips]{graphicx} 
\usepackage{amsmath} 
\usepackage{amssymb} 
\usepackage{amsbsy} 
\usepackage{amsfonts} 
\usepackage{array} 
\usepackage{verbatim}
\usepackage{floatflt} 

\newtheorem{remark}{Osservazione}[section] 
 
\title{Libreria per elementi finiti isoparametrci monodimensionali}

\author{Marco Restelli}

\date{}

\begin{document}

\maketitle

\section{Premessa}

Nella presente guida viene fissata la notazione utilizzata in tutte le
funzioni del pacchetto EF1D. In particolare, vengono fissate tutte le
convenzioni relative alla forma delle condizioni al contorno e alla
forma del problema differenziale, seguendo quella che viene usualmente
detta \emph{forma conservativa} del problema.

\begin{remark}
Si noti che, nell'usare il pacchetto EF1D, \`e fondamentale attenersi
alle convenzioni descritte in questa guida.
\end{remark}

\section{Notazione}

Sia $I$ un intervallo aperto di estremi $a$ e $b$ (con $a<b$).
Indichiamo con $\Gamma$ la frontiera di $I$: $\Gamma = \{a\}\cup\{b\}$
e con ${\bf n}$ il versore uscente su $\Gamma$: ${\bf n}(a)=-1$ e
${\bf n}(b)=1$. Dato il campo ${\bf a}$, seguendo~\cite{Hughes}
definiamo 
\[
\begin{array}{rcll}
a_n & = & {\bf n}\cdot{\bf a} & {\rm (flusso\,\,normale)} \\[2mm]
a_n^+ & = & (a_n+|a_n|)/2 & {\rm (flusso\,\,normale\,\,uscente)} \\[2mm]
a_n^- & = & (a_n-|a_n|)/2 & {\rm (flusso\,\,normale\,\,entrante)}.
\end{array}
\]
Al fine di specificare le condizioni al contorno del problema,
introduciamo inoltre una partizione $\Gamma_g$, $\Gamma_h$ di $\Gamma$
tale che $\Gamma = \Gamma_g \cup \Gamma_h$ e $\emptyset = \Gamma_g \cap
\Gamma_h$. Questo permetter\`a di specificare condizioni di prima
specie su $\Gamma_g$ e di seconda specie su $\Gamma_h$.

\begin{remark}
\`E possibile avere $\Gamma_g=\emptyset$, nel qual caso sar\`a
$\Gamma_h=\Gamma$, o viceversa.
\end{remark}

\section{Il caso stazionario}
Il problema di diffusione-trasporto-reazione nella formulazione
conservativa \`e il seguente:
\begin{equation}
\label{eq:sistema}
\left \{
\begin{array}{lrl}
\displaystyle -\frac{d}{dx} \left( \mu \frac{du}{dx} \right) + 
\frac{d}{dx} \left( {\bf a} u \right) + \sigma u = f 
& {\rm in} & I \\[2mm]
\displaystyle u = g & {\rm su} & \Gamma_g\\[2mm]
\displaystyle \left ( -\mu \frac{du}{dn} + a_n^- u \right) 
- \gamma u = r
& {\rm su} & \Gamma_h.
\end{array}
\right.
\end{equation}
\begin{remark}
La forma della condizione al contorno prescritta su $\Gamma_h$
necessita qualche commento.
\begin{itemize}
\item Essendo ${\bf n}$ la normale uscente rispetto a $I$,
$\frac{du}{dn}$ rappresenta la derivata di $u$ nella direzione uscente
rispetto all'intervallo, ovvero
\[
\left.\frac{du}{dn}\right|_a  =  - \left.\frac{du}{dx}\right|_a,
\qquad
\left.\frac{du}{dn}\right|_b  = \left.\frac{du}{dx}\right|_b.
\]
L'uso di $\frac{du}{dn}$ nella~(\ref{eq:sistema})$_3$ al posto di
$\frac{du}{dx}$ \`e motivato dal fatto che, nel suo insieme, la
quantit\`a $-\mu\frac{du}{dn}$ ha l'importante significato fisico di
\emph{flusso diffusivo sul bordo di $I$, positivo se uscente}.
\item Nel suo insieme, la quantit\`a $a_n u$ ha l'importante
significato fisico di \emph{flusso convettivo sul bordo di $I$,
positivo se uscente}. L'uso di $a_n^-$ nella~(\ref{eq:sistema})$_3$
comporta dunque che:
\begin{itemize}
\item dove ${\bf a}$ \`e entrante, essendo $a_n^-=a_n$, viene
prescritto il flusso totale diffusivo e convettivo
\[
-\mu\frac{du}{dn} + a_n u,
\]
\item dove ${\bf a}$ \`e uscente, essendo $a_n^-=0$, viene
prescritto il solo flusso diffusivo
\[
-\mu\frac{du}{dn}.
\]
\end{itemize}
\item La definizione del coefficiente $\gamma$ \`e tale che,
tipicamente, si avr\`a $\gamma\geq0$.
\end{itemize}
\end{remark}

Introducendo ora un opportuno spazio funzionale $V$, il
problema~(\ref{eq:sistema}) pu\`o essere formulato in senso debole nel
modo seguente: \\
trovare $u \in V$ tale che
\begin{equation}
\label{debole}
\begin{array}{l}
\displaystyle
\int_I \mu \frac{du}{dx} \, \frac{d\phi}{dx} \,dx -
\int_I u \, {\bf a} \frac{d\phi}{dx} \,dx +
\int_I \sigma u  \phi \,dx +
 a_n^+ u \phi|_a + a_n^+ u \phi|_b +
 \gamma u \phi|_a + \gamma u \phi|_b = \\[4mm]
\displaystyle
\qquad \qquad\int_I f\phi \,dx -
 r \phi|_a -r\phi|_b \quad \qquad \qquad\forall \phi \in V
\end{array}
\end{equation}

Definito ora lo spazio di dimensione finita $V_h \subset V$ tale che
${\rm dim}(V_h)=N_h$, indicata con $\{\phi_i \}_{i=1}^{N_h}$ una base
di $V_h$, e posto $u_h = \sum_{j=1}^{N_h}u_j\,\phi_j$, otteniamo il
problema discertizzato\\
trovare ${\bf u}\in \mathbb{R}^{N_h}$ tale che
\begin{equation}
\label{deboledisc}
\begin{array}{r}
\left[ 
\underbrace{\int_I \mu \frac{d\phi_i}{dx} \, \frac{d\phi_j}{dx} \,dx
           }_{{\bf K}_{ij}} \, \,
\underbrace{-\int_I {\bf a} \frac{d\phi_i}{dx} \, \phi_j \, dx
            + a_n^+ \phi_i \phi_j|_a + a_n^+ \phi_i \phi_j|_b 
	   }_{{\bf C}_{ij}}\, \, 
\underbrace{+\int_I \sigma \phi_i  \phi_j \,dx 
           }_{{\bf S}_{ij}} \right .\\[2mm]
\qquad \left.
\underbrace{+ \gamma \phi_i \phi_j|_a + \gamma \phi_i \phi_j|_b
           }_{{\bf R}_{ij}}
\right] u_j =  
\underbrace{\int_I f\phi_i \,dx }_{{{\bf b}_v}_{i}} \, \,
\underbrace{- r \phi_i|_a -r \phi_i|_b}_{{\bf r}_{i}}
\end{array}
\qquad i = 1, \hdots, N_h.
\end{equation}

La libreria EF1D fornisce le funzioni per la costruzione delle matrci 
${\bf K}, \, {\bf C}, \,{\bf S}$ e ${\bf R}$ e dei vettori ${\bf b}_v$ e
${\bf r}$ nella~(\ref{deboledisc}). 
In aggiunta alle formulazioni base, per quanto riguarda ${\bf C}$ \`e 
possibile applicare varie tecniche di stabilizzaione numerica 
(vedi~\cite{Quarteroni}), mentre per ${\bf S}$ \`e possibile costruire 
la versione con condensazione ({\em lumping}).

Posto ${\bf A}={\bf K} + {\bf C} + {\bf S} + {\bf R}$ e
${\bf b}={\bf b}_v + {\bf r}$, otteniamo il sistema lineare
\begin{equation}
\label{sistAub}
{\bf A}\,{\bf u} = {\bf b}
\end{equation}
che pu\`o essere risolto dopo l'introduzione delle eventuali condizioni
al bordo assegnate per $\Gamma_g$.

\section{Il caso non stazionario}

Dato l'intervallo temporale $I_T=(0,T)$, consideriamo il problema 
seguente:
\begin{equation}
\label{eq:sistema_nstaz}
\left \{
\begin{array}{lrl}
\displaystyle 
\rho \frac{\partial u}{\partial t}
- \frac{\partial}{\partial x} 
           \left( \mu \frac{\partial u}{\partial x} \right) + 
  \frac{\partial}{\partial x} 
           \left( {\bf a} u \right) + \sigma u = f & {\rm in} &
  I\times I_T \\[2mm]
  u = u_0 & {\rm in} & I \times \{0\} \\[2mm]
  u = g & {\rm su} & \Gamma_g \times I_T \\[2mm]
  \displaystyle \left ( -\mu \frac{\partial u}{\partial n} 
                        + a_n^- u \right) - \gamma u = r
  & {\rm su} & \Gamma_h \times I_T,
\end{array}
\right .
\end{equation}
di cui la (\ref{eq:sistema}) rappresenta il caso limite per
$\frac{\partial u}{\partial t}=0$.

Procedendo per la discretizzazione in spazio come nel caso stazionario
e introducendo la matrice di massa
$$
{\bf M}_{ij}= \int_I \rho\, \phi_i \, \phi_j \,dx
$$
otteniamo il problema semidiscretizzato:
\begin{equation}
\label{sistNstaz}
{\bf M}\,\dot{{\bf u}}(t) + {\bf A}\,{\bf u}(t) = {\bf b}(t),
\qquad t\in I_T.
\end{equation}

L'applicazione del $\theta$--metodo (vedi~\cite{Wood})
per la discretizzazione in tempo del problema~(\ref{sistNstaz}) conduce 
all'equazione seguente:
\begin{equation}
\label{thetamet}
\underbrace{({\bf M} + \Delta t\, \theta\, {\bf A})
           }_{{\bf B}}{\bf u}^{n+1} =
\underbrace{({\bf M} - \Delta t \, (1-\theta)\, {\bf A})
           }_{{\bf Z}}{\bf u}^{n} + 
	   \theta\,{\bf b}^{n+1} + (1-\theta){\bf b}^n
\end{equation}
che, nota ${\bf u}^n$, pu\`o essere risolta per ${\bf u}^{n+1}$.

La libreria EF1D fornisce le funzioni per il calcolo di {\bf M}
nelle versioni completa e con condensazione nonch\'e per la 
costruzione della matrice di iterazione ${\bf B}^{-1}{\bf Z}$ e la 
risoluzione della~(\ref{thetamet}).


\begin{thebibliography}{99}
\bibitem{Hughes}T.J.R.~Hughes, G.~Engel, L.~Mazzei, M.G.~Larson:
{\em The Continuous Galerkin Method Is Locally Conservative},
Journal of Computational Physics 163, 467--488 (2000)
\bibitem{Quarteroni}A.~Quarteroni,
{\em Modellistica Numerica per Problemi Differenziali},
Springer Verlag
\bibitem{Wood}W.L.~Wood,
{\em Practical Time-stepping Schemes},
Clarendon Press, (1990)
\end{thebibliography}

\end{document}
